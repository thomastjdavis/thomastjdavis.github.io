\documentclass[fleqn]{article}
\author{Calculus - TAC Shanghai}

\usepackage{amsmath}
\title{Chapter 2: Limits Quiz Solution}
%\author{Thomas Davis}
\date{November 4, 2020}
\setlength{\mathindent}{0pt}

\begin{document}

\maketitle
\textbf{INSTRUCTIONS:} please solve the following limit problems. Some limits may not exist.
Show \textit{ALL} your work, either on this paper or on a separate sheet of paper.


\[1.\lim_{h\to0}\frac{\big(h-1\big)^4-1}{h} \]
Evaluating leads to $\frac{0}{0}$, so we need to manipulate the expression:
\[= \lim_{h\to0}\frac{(h^2-2h+1)^2-1}{h}=\lim_{h\to0}{\frac{\big[(h^2-2h+1)(h^2-2h+1)\big]-1}{h}}\]
\[\lim_{h\to0}\frac{\big(h^2(h^2-2h+1)-2h(h^2-2h+1)+1(h^2-2h+1)\big)-1}{h}\]
\[=\lim_{h\to0}\frac{\big(h^4-4h^3+6h^2-4h+1\big)-1}{h}=-4\]

\[2.\lim_{x\to7}\frac{\sqrt{x^2-24}-4}{\sqrt{x^2+4}-1}\]
Both numerator and demoninator are non-zero at $x=7$, so we can directly evaluate:

\[\lim_{x\to7}\frac{\sqrt{x^2-24}-4}{\sqrt{x^2+4}-1}=\frac{\sqrt{(7)^2-24}-4}{\sqrt{(7)^2+4}-1}\]
\[=\frac{1}{\sqrt{53}-1} \text{ or } \frac{\sqrt{53}+1}{52}\]


\[3.\lim_{z\to-2}\frac{\frac{1}{z-2}-\frac{1}{3z+2}}{z+2}\]
Evaluating leads to $\frac{0}{0}$, so we need to manipulate the expression:
\[=\lim_{z\to-2}\frac{1}{z+2}\cdot\big(\frac{1}{z-2}-\frac{1}{3z+2}\big)\]

\[=\lim_{z\to-2}\frac{1}{z+2}\cdot\frac{(3z+2)-(z-2)}{(z-2)(3z+2)}\]
\[=\lim_{z\to-2}\frac{1}{z+2}\cdot\frac{2z+4}{(z-2)(3z+2)}=\lim_{z\to-2}\frac{1}{z+2}\cdot\frac{2\cdot(z+2)}{(z-2)(3z+2)}\]
\[\lim_{z\to-2}\frac{2}{(z-2)(3z+2)}\]

Both numerator and demoninator are non-zero at $z=-2$, so we can directly evaluate:
\[\lim_{z\to-2}\frac{2}{(z-2)(3z+2)}=\frac{2}{((-2)-2)(3(-2)+2)}\]
\[=\frac{1}{8}\]

\[4.\lim_{\vartheta\to0}\frac{\vartheta^2}{\tan^2 5\vartheta}\]
\textit{Hint}: You should use the fact $ \lim_{\theta\to0}\frac{\sin\theta}{\theta}=1 $ twice.

Directly evaluating gives $\frac{0}{0}$, so we need to manipulate the expression.

\[\lim_{\vartheta\to0}\frac{\vartheta^2}{\tan^2 5\vartheta}=\lim_{\vartheta\to0}\frac{\vartheta^2}{1}\cdot\frac{\cos^2(5\vartheta)}{\sin^2(5\vartheta)}=\lim_{\vartheta\to0}\frac{\cos^2(5\vartheta)}{1}\cdot\frac{\vartheta^2}{\sin^2(5\vartheta)}\]
\[=\big[\lim_{\vartheta\to0}\frac{\cos^2(5\vartheta)}{1}\big]\cdot\big[\lim_{\vartheta\to0}\frac{\vartheta^2}{\sin^2(5\vartheta)}]=1\cdot\lim_{\vartheta\to0}\frac{\vartheta^2}{\sin^2(5\vartheta)}=\bigg[\lim_{\vartheta\to0}\frac{\vartheta}{\sin(5\vartheta)}\bigg]^2\]
Now, we need to remember the hint from earlier:
\[\lim_{\theta\to0}\frac{\sin(\theta)}{\theta}=1\]
Or, when $a$ is a real number:
\[\lim_{\theta\to0}\frac{\sin(a\theta)}{\theta}=a\]
We have instead:
\[\lim_{\vartheta\to0}\frac{\vartheta}{\sin(5\vartheta)}=\lim_{\vartheta\to0}\frac{1}{\frac{\sin(5\vartheta)}{\vartheta}}=\frac{1}{5}\]
\[\implies\lim_{\vartheta\to0}\frac{\vartheta^2}{\tan^2 5\vartheta}=\bigg[\lim_{\vartheta\to0}\frac{\vartheta}{\sin(5\vartheta)}\bigg]^2=\bigg[\frac{1}{5}\bigg]^2=\frac{1}{25}\]


\[5.\lim_{x\to\infty}\frac{x^4+7x}{x^3-2}\]
On top the highest power is $x^4$, and on bottom the highest power is $x^3$. Factoring out the highest power on top and bottom yields:
\[\lim_{x\to\infty}\frac{x^4+7x}{x^3-2}=\lim_{x\to\infty}\frac{(x^4)(1+7\frac{1}{x^3})}{(x^3)(1-\frac{2}{x^3})}\]
\[\lim_{x\to\infty}\frac{x(1+7\frac{1}{x^3})}{1-\frac{2}{x^3}}=\lim_{x\to\infty}x=\infty\]


\[6.\lim_{x\to -\infty}\frac{5e^{2x}+10e^{-3x}}{2e^{5x}+37e^{-3x}}\]
When $x\to-\infty, e^{-x}$ dominates and $e^{x}\to0$. This means we need to divide by the largest power of $e^{-x}$ to determine the limit.\linebreak
Doing so:
\[\lim_{x\to -\infty}\frac{5e^{2x}+10e^{-3x}}{2e^{5x}+37e^{-3x}}=\lim_{x\to-\infty}\frac{(e^{-3x})(5e^{5x}+37)}{(e^{-3x})(2e^{8x}+37)}\]
\[=\lim_{x\to-\infty}\frac{5e^{5x}+37}{2e^{8x}+37}=\frac{10}{37}\]

7. Given that $3+2x\leq f(x)\leq x-1$ for all $x$, determine the value of $\lim_{x\to-4}f(x)$. \linebreak

Notice that $3+2(-4)= -5 = (-4)-1$, and both the left and right hand sides are continuous. We can then apply the Squeeze/Sandwich Theorem:
\[\lim_{x\to-4}3+2x\leq\lim_{x\to-4} f(x)\leq\lim_{x\to-4} x-1\]
\[\implies -5\leq\lim_{x\to-4} f(x) \leq -5\implies \lim_{x\to-4}f(x)=-5\]

8. Describe what it means for a function to be \textit{continuous} on an interval. \linebreak

\textit{Note:} I accepted anything from "$\lim_{x\to a}f(x) =f(a)$ on every point on an interval", to "the graph can be drawn with no holes.

9. On the back of this sheet of paper, draw a function with the following criteria:

\begin{itemize}
    \item $\lim_{x\to3}f(x)=5 \text{ but } f(3)=-1$
    \item $\lim_{x\to-\infty}f(x)=0 \text{ and } \lim_{x\to\infty}f(x)=1$
\end{itemize}

\textit{Note:} This function could have many forms, but needs to be discontinuous at $x=3$, and match the two limits at infinity.

Use the following information for question 10:

Georgia goes to the doctor, and her doctor informs her that she now has a disease which lowers her body's levels of chemical X.
In order to stabilize her health, Dr. Chen perscribes a drug, MiracleX which helps restore Georgia to health.
Based on Dr. Chen's medical knowledge, he predicts that Georgia's levels of chemical X will follow the following function:


\begin{equation*}
    X(t)=\begin{cases}
        20 \text{ ml} \quad & t\leq 0 \\
        \frac{60}{2+e^{-.5t}} \text{ ml} \quad & t > 0\\ 
         \end{cases}
\end{equation*}
where $t$ is measured in days, and $t=0$ is the first day Georgia takes MiracleX.


Answer the following:


10. If Georgia continues to take the medicine over time, will her body's levels of chemical X stabilize? 
If she needs 27.5 ml to be considered healthy again, will her body reach that? \linebreak   

\textit{Note:} This was not super clear of a problem - if I use this again I will reword it to make more sense.
We need to look at $\lim_{t\to\infty}\frac{60}{2+e^{-.5t}}$. \linebreak
\textbf{Way 1}:
We know that $\lim_{t\to\infty}e^{-.5t}=0$, so we can apply the limit quotient and sum rules:
\[\lim_{t\to\infty}\frac{60}{2+e^{-.5t}}=\frac{\lim_{t\to\infty}60}{\lim_{t\to\infty}2+\lim_{t\to\infty}e^{-.5t}}\]
\[=\frac{60}{2+0}=30\]
\textbf{Way 2}:
Multiplying top and bottom by $e^{.5t}$, we have:
\[\lim_{t\to\infty}\frac{e^{.5t}}{e^{.5t}}\cdot\frac{60}{2+e^{-.5t}}=\frac{60e^{.5t}}{2e^{.5t}+1}\]
If we factor the largest factor of $e^{t}$ on top and bottom like earlier, we obtain the same result.\linebreak

Since $30>27.5$, Georgia will be considered healthy again.

\end{document}